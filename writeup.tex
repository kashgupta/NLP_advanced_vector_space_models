\documentclass[12pt]{article}
\setlength{\oddsidemargin}{0in}
\setlength{\evensidemargin}{0in}
\setlength{\textwidth}{6.5in}
\setlength{\parindent}{0in}
\setlength{\parskip}{\baselineskip}

\usepackage{amsmath,amsfonts,amssymb}

\title{Your Project Title}

\begin{document}

CIS530 Spring 2018\hfill Assignment 4\\
Firstname Lastname (MM/DD)

\hrulefill

\section{Sparse Vector Representations}

We tried X variations of a sparse vector representation. The variations we tried were:

\begin{itemize}
\item ...
\end{itemize}

We used XYZ as the clustering algorithm. 
% If you experimented with multiple clustering algorithms, include that here.

We compare the impact of our model and clustering choices by evaluating their performance over the dev set (Table \ref{tab:sparseresults}):

% Your table may have a different number of rows/columns based on the experiments you ran
\begin{table}[]
\centering
\begin{tabular}{lll}
Vector Space Model  & KMeans & My Clustering Method 2 \\
Baseline (provided) &        &                        \\
My Model 1          &        &                        \\
My Model 2          &        &                       
\end{tabular}
\caption{Paired F-Score on the dev set by different vector space models and clustering algorithms.}
\label{tab:sparseresults}
\end{table}

\section{Dense Vector Representations}

We tried X variations of the dense vector representation. The variations we tried were:

\begin{itemize}
\item ...
\end{itemize}

We used XYZ as the clustering algorithm. 
% If you experimented with multiple clustering algorithms, include that here.

We compare the impact of our model and clustering choices by evaluating their performance over the dev set (Table \ref{tab:denseresults}):

% Your table may have a different number of rows/columns based on the experiments you ran
\begin{table}[]
\centering
\begin{tabular}{lll}
Dense Model  & KMeans & My Clustering Method 2 \\
Word2Vec (provided) &        &                        \\
My Model 1          &        &                        \\
My Model 2          &        &                       
\end{tabular}
\caption{Paired F-Score on the dev set by different dense vector space models and clustering algorithms.}
\label{tab:denseresults}
\end{table}

\section{Comparison}

% This is just an example of what a comparison might look like -- feel free to do additional analyses 
Overall, our XYZ model, a (sparse/dense) model, performs best. 

In general, we can compare dense and sparse models by looking at instances where one does well and the other fails.

Examples of target words where the dense model scores high and the sparse model scores low are... (describe what they have in common, hypothesize why this might be the case)

Examples of target words where the sparse model scores high and the dense model scores low are... (describe what they have in common, hypothesize why this might be the case)



\section{(Optional) Choosing K}

We designed a model that chose the best K as follows...

Our results on the \texttt{test_nok_input.txt} data are given in (table below)...

\end{document}